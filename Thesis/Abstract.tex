In this thesis, we develop and apply a novel algorithm to understand 
the evolution of stellar discs in $\Lambda$CDM cosmological haloes.
Three main scientific areas are addressed: the effects of evolving stellar
discs on their host haloes, understanding bar formation in cosmological settings,
and the formation of vertical disc structure in response to the host halo.

First, we find that the presence of central concentrations of baryons in dark 
matter haloes enhances adiabatic contraction causes an overall modest 
reduction in substructure. Additionally, the detailed evolution of stellar discs 
is important for the inner halo density distribution. However,
properties of the halo such as the subhalo mass function are broadly unaffected
by the evolution of a realistic stellar disc.


Next, we find that stellar bars invariably form in Milky Way-like galaxies. The 
strength of these bars is less dependent on properties like the dynamical
temperature of the disc in a cosmological setting. Instead, the disc thickness
plays a leading role in determining the overall bar strength. Our discs
undergo notable buckling events, yielding present-day pseudobulge-disc-halo
systems. We show that these are qualitatively similar to the observed Milky Way.

Finally, we show that a wide variety of vertical structure forms when 
stellar discs are embedded in cosmological haloes. We further show that through
a variety of mechanisms, similar vertical structure is excited. By examining
twelve simulations of discs in cosmological haloes, we show that the Sgr dSph
need not be as massive as $10^{11}\,M_\odot$ to be consistent with observed
vertical structure in the Milky Way. In fact, the recent buckling of the Milky Way's
bar and its present interaction with the LMC are more likely culprits for some of
the observed structure in the Milky Way's thin disc.