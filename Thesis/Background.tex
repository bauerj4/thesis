\chapter{A Dynamical Recipe for Cosmological Disks}\label{ch:background}

This chapter provides an overview of the background theory needed to understand subsequent chapters. In Chapter~\ref{ch:introduction}, we had a lot of discussion about simulations of galaxies being critical to having a theoretical understanding of cosmology and galaxy formation. The word ``simulation" was used without fully contextualizing its significance and meaning. We provide that context in \S\ref{sec:motivation}, where we describe what is actually being done when a simulation is run. In \S\ref{sec:galaxy_ics}, we talk about how to apply this theory to study the evolution of equilibrium galaxies. We also spoke at length about $\Lambda$CDM cosmology and how it is critical to explain the evolution of galaxies on a global scale. We talk about cosmology in detail in \S\ref{sec:cosmology}, and explain how this specifies an extended view of the discussion in \S\ref{sec:motivation}. Lastly, we talk about various techniques relied upon in the analysis of simulation in \S\ref{sec:analysis_of_sims}. This includes how we identify cosmic substructure and the techniques we use for analyzing the time-evolution of disk-derived quantities. This chapter summarizes the very basics of the models used in this thesis. It is by no means a comprehensive account of galactic dynamics. In the course of discussion, we will point the reader to more detailed accounts of the topics being discussed.

\section{Physical Motivation of Modeling} \label{sec:motivation}

The goal of this section is to convey to the reader what we mean by a simulation of a galactic system. 

\subsection{Characterizing of Self-Gravitating Systems}

The baryonic mass of the Milky Way is largely concentrated in its stars. To first order, the dynamical behavior of the Milky Way is determined by stellar material and dark matter \citep{BM}. While the evolution of gas is governed by the Euler equations with an equation of state, modelling in galactic dynamics requires that we understand how stars and dark matter behave. 

Unlike gas, which can exhange thermodynamic energy with itself, stars and dark matter interact only through gravity. Unlike the electromagnetic force, gravity is a long range force. In fact, the majority of the contribution to the forces on stars in galaxies comes from far outside their immediate neighborhoods \citep{BT}. This has a number of interesting implications. In classical thermodynamics, we can derive the macroscopic 

\subsection{The Collisionless Boltzmann Equation}
\subsection{Numerical Solutions}

\section{Phase Space, Equilibrium, and Initial Conditions} \label{sec:galaxy_ics}
\subsection{Jeans Modeling}
\subsection{DF-based Models and the Strong Jeans Theorem}
\subsection{Action-Angle Variables}
\subsection{DFs as Functions of Actions}

\section{Cosmology and Implications for Galaxy Studies} \label{sec:cosmology}
\subsection{Basic FRW Cosmology}
\subsection{Extension of Numerical Methods}
\subsection{Sampling Initial Conditions for Cosmological Simulations}


\section{Simulation Analysis: Paradigms and Tools} \label{sec:analysis_of_sims}
\subsection{Cosmological Substructure and Halo Triaxiality}
\subsection{Identifying Substructure in Simulations}
\subsection{WKB Wave Analysis}
\subsection{Time Series Filtering}
\subsection{MCMC}
\bibliographystyle{apalike}
\bibliography{bibliography_background}


