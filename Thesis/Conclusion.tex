\chapter{Summary and Conclusions}\label{ch:Conclusion}

In this chapter, we summarize the  main findings of this thesis and talk about future work.


\section{Interactions Between Stellar Disks and Dark Matter Halos}


\subsection{Summary}

One of the key contributions of this thesis is the understanding of how the dynamic nature of stellar disks affects halo properties. Although the material in Chapter~\ref{ch:paper_i} is primarily focused on introducing a method used by our group, its key scientific contributions focus on this point.

We found that modelling the disk in a dynamic fashion is tremendously important for the inner halo. Specifically, we found the formation of a bar led to a more concentrated halo mass distribution.

The intermediate-to-outer halo was less effected by the dynamics of the inner halo. While some differences between our disk potential models were observed in the halo mass functions, these differences were minor. The details of how satellites are disrupted appear to be at least weakly dependent on how the disk is modelled, and stream studies should consider as realistic of a disk as possible.

\subsection{Future Work}

\section{Bar Formation in Cosmological Stellar Disks}
\subsection{Summary}

\subsection{Future Work}
\section{Vetical Structure in Cosmological Stellar Disks}
