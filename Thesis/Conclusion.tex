\chapter{Summary and Conclusions}\label{ch:conclusion}
\newpage


In this chapter, we summarize the  main findings of this thesis and talk about future work.


\section{Disk-Halo Interactions}


\subsection{Summary}

One of the key contributions of this thesis is the understanding of how the dynamic nature of stellar disks affects halo properties, as studied in Chapter~\ref{ch:paper_i}. Some halo properties, such as the distribution of mass in the inner halo, depend a lot on how the disk is modelled. This is very evident in the differences in adiabatic contraction between the rigid and live disks. Specifically, we find the formation of a bar led to a more concentrated halo mass distribution. On the whole, these results suggest that if one is interested in the Core-Cusp problem, this should be studied with fully live models of stellar disks.

The intermediate-to-outer halo is less effected by the dynamics of the inner halo. While some differences between our disk potential models were observed in the halo mass functions, these differences were minor. The details of how satellites are disrupted appear to be at least weakly dependent on how the disk is modelled, and stream studies should consider as realistic of a disk as possible. This last aspect of our work has informed some design decisions by others, namely in the work of \citet{read_2019}. If one is just interested in subhalo population statistics, it is sufficient to model baryons as fixed potentials in a pure dark matter halo.

The global picture painted by our work is that modelling the disk as a rigid body does well in determining how the disk settles (i.e. its angular motion), a result consistent with \citet{DubinskiKuijkenRigidDisks}. It appears that the formation of a bar in the live disk is the most substantial difference between the rigid and live disks' effects on the inner halo, a difference that only becomes significant \textit{after} the the live disk



%Beyond the effect of the disk on the halo, we find that our rigid disk approximation is quite realistic in modelling the angular evolution of the disk. This allows us to set up near-equilibrium cosmological systems without the use of MHD simulations. Since the work in this thesis focuses on the underlying theoretical mechanisms driving the evolution of galaxies in cosmological environments, we only need a good representation of the true systems.

\subsection{Insertion Scheme and Future Work} \label{ssec:conclusions_insertion}

Our algorithm makes optimizations over previous disk insertion schemes. The optimizations we made are by no means exhaustive, and a detailed discussion of potential optimizations is given in Chapter~\ref{ch:implementation}. We focus here on scientific concerns with our algorithm, rather than performance optimizations. 

The scheme as specified has the undesirable property of adding angular momentum and mass to the halo. This happens when the disk grows, and we fail to extract mass and angular momentum from the halo\footnote{Note that dark matter particles dark matter only simulations include the baryon mass of the Universe.}. A potential improvement which extracts mass and angular momentum from the halo particles during the disk growth phase would alleviate these concerns. This may have a sizable impact on our conclusions about the disk's effect on adiabatic contraction, since both the disk and halo are dominant in the inner galaxy.

We might also consider a scheme which initializes the disk with some gas component. \citet{deg_2019} modified \textsc{GalactICS} to allow for the inclusion of gas. This, combined with modifying halo particle masses, should allow us to more accurately compare our results to fully hydrodynamical \textit{ab initio} simulations.

A valid criticism of our disk insertion technique is that all of the stars are initialized simultaneously. This is unrealistic, and allowing stars to gradually form out of the rigid disk potential might show a more complicated dependence on the initial conditions parameter space. In fact, \citet{aumer_2016} grow disks in a similar fashion and find that slowly growing the disk along with gas physics impacts important properties like bar formation. In some cases it may even shut off bar formation, which we believe would substantially impact our conclusions about adiabatic contraction.

Even with the current algorithm, there is more we can do on the study of disk-halo interactions. A possible project would be to study the effect of varying disk properties in the same halo. We might also consider adding a spheroidal bulge component to see how high baryonic mass concentrations change our results. We might also look for more detailed evidence of the halo's response to the disk in our existing simulations in the form of wakes. This kind of response was explored in one of our toy models in Chapter~\ref{ch:paper_iii}, and it seems natural to extend this analysis to our cosmological simulations.


\section{Bar Formation}
\subsection{Summary}

We discuss our contributions to understanding how bars form in stellar disks in Chapter~\ref{ch:paper_ii}. The paper unifies previous work on disk insertion schemes by viewing them through a lens of a common parameter space. Through placing our models and the models of \citet{YurinSpringelStellarDisks} in this space, we are able to contextualize the very large bars formed in their study. In particular, we show that a commonly discounted variable, the disk thickness, is a primary driver of bar strength in a cosmological setting. We show that, in general, the length of a bar at present day for cosmological stellar disks is largely a function of the initial disk thickness. This effect is not numerical, as we show through our application of the \textsc{AGAMA} code \citep{vasiliev_2018}. 

The importance of being cautious about the choice of gravitational softening length is also illustrated. Since disk thickness appears to play a major role in bar formation in $\Lambda$CDM, we must be careful that we are resolving our disks' vertical structure.


On the whole, we show that it is difficult to suppress the formation of bars in a cosmological setting. While the bar more closely resemble the Milky Way's when we start with a thinner disk, we find strong bars form nonetheless. As such, there is still a discrepancy in accounting for the observed bar fraction without invoking classical bulges or non-collisionless phenomena. Our work has proved useful in illustrating this, and we are credited for furthering discussion of this issue in \citet{sellwood_2019}.


We manage to bolster this result in Chapter~\ref{ch:paper_iii}, where we ran twelve simulations in four halos. Although all of the halos had the same total mass as the Milky Way, they varied widely in terms of their other properties. Despite the differences in the halos, the bars formed by embedded disks were quite similar. Additionally, there is a nontrivial dependence on Toomre $Q$ in terms of when the bar buckles, but the final bar strength depends very weakly on this parameter, contrary to \citet{YurinSpringelStellarDisks}. Rather, the trend \citet{YurinSpringelStellarDisks} sees in $Q$ is likely due to the presence of their bulge simulations, which would increase $Q$ as well as provide a stabilizing central mass.


\subsection{Further Exploration}


In Chapter~\ref{ch:paper_iii}, bar formation is not the focus of the study. Now that we have a sample of bars forming in different cosmological environments, we can position the disks in the parameter space described in Chapter~\ref{ch:paper_ii}. A proper synthesis with previous literature on this topic should be performed.

An interesting feature of our simulations is the existence of ``banana bars" in some of the simulations. Understanding why we get banana bars in some simulations versus a more traditional X-shaped pattern might provide an observational constraint for the Milky Way. This topic should be explored in more detail.

It would also be interesting to see how implementing a hydrodynamical, growing disk described in \S\ref{ssec:conclusions_insertion} would change our results. An interesting conclusion from \citet{aumer_2016} was that bar formation could be suppressed by GMCs. It is worth noting that these simulations also included a compact bulge, and the result might be different for a pure disk-halo system in a fully cosmological environment. The author's view is that moving in this direction is the next logical step for disk-insertion schemes. 



\section{Vertical Structure}
\subsection{Summary}

The main focus of Chapter~\ref{ch:paper_iii} is how disks form vertical structure in $\Lambda$CDM environments. We showed that there are a wide variety of Milky Way-mass halos that form vertical structure consistent with structure observed in the solar neighborhood. In particular, we showed that this structure can be formed without a massive satellite encounter in a cosmological setting. This position is bolstered by the fact that we see Monoceros-like structure in all twelve of our simulations.

We also find that consistent perpendicular torques on the disk are required to explain features in the outer disk. It is unlikely that substructures far below the $10^{11}$ solar mass regime are responsible for stars stripped to higher latitudes. Global tides from disk-halo axis misalignment seem to be the most effective way to create these populations of stars. We therefore suggest that streams originating from the outer disk may primarily depend on the Milky Way's interaction with the LMC. A massive merger like the one with the LMC can cause consistent perpendicular torques its own tides or by disturbing the existing smooth Milky Way halo mass. 

\subsection{More Accurate Milky Way Comparisons}

A key flaw in this work is that we had no LMC analogues. An interesting experiment would be to compare the vertical structure caused by a future LMC merger and the vertical structure generated when a disk is misaligned in a triaxial halo. On the whole, future work should proceed in the same way as the work in Chapter~\ref{ch:paper_ii}. We should seek to relate the halo environments to a common parameter space.

We also might want to compare our simulations in the solar neighborhood to Gaia data through simulation-generated mock catalogues. Although not presented in this thesis, we have developed a code which allows us to generate initial conditions that are selectively upsampled in a specified annulus. In isolated galaxy simulations, the upsampled annulus remains mostly uncontaminated at late times. By resampling the solar neighborhood, and by using simulation-generated mock catalogues, we can compare our simulations more directly with observations.

\section{Closing Remarks}

%This thesis considered a very specific approach to understanding a very specific concept: the collisionless evolution of systems with stellar disks and dark matter halos. It is tremendously humbling to note that we never even considered gas dynamics, a cosmology other than $\Lambda$CDM, or even systems with a strong, pre-existing bulge. In the end, we were able to make some elucidating comments on the nature of stellar dynamics in $\Lambda$CDM cosmology, but no more. 

This work presents an improved disk insertion scheme with applications to Galactic astronomy. We have successfully shown that we can set up and understand disks inserted into cosmological halos. Future work is suggested, and we have no shortage of potential projects to improve our scheme. The long-term goal of this work is to bridge the gap with \textit{ab initio} simulations, and we have presented a clear roadmap for doing so.

As simulation resolutions and computing power improve, we may be forced to revisit the fundamentals at the beginning of Chapter~\ref{ch:background}. Simulation codes may need to be redesigned to accommodate ever-increasing needs to relate them to observations. Our hope is that disk insertion schemes will be at the heart of these discussions as a valuable tool in understanding observational data.

Our work represents a tangible leap forward, however small, in the field of Galactic dynamics.  We hope that our work continues to inspire new developments, new comparisons, and more questions. In particular, we hope to see our work mature, and possibly tell us about the nature of the dark Universe in which we live. This is an incredibly exciting time for the field, one where we are proud to have played a small role.



\bibliographystyle{apalike}
\bibliography{bibliography_conclusions.bib}