\section{Complexity and Performance}\label{sec:EmbeeComplexity}

\subsection{Definition of Terms}\label{sec:termDefn}

    The following terms...:

    \begin{Ventry}{\boldmath $\operatorname{arity}(r_i)$ \unboldmath}

        \boldmath \item[$scope$] \unboldmath
        The maximum number of objects...

        \boldmath \item[$R$] \unboldmath
        The number of relations...

        \boldmath \item[$r_i$] \unboldmath
        The $i^\text{th}$ relation in the specification, $1 \le i \le R$.

        \boldmath \item[$\operatorname{arity}(r_i)$] \unboldmath
        The arity of relation $r_i$...

        \boldmath \item[$N$] \unboldmath The total number...

        Given the calculated arities of a particular specification's relations, and the scope at
        a specific breakpoint, Equation~\ref{equation:N} can be used to determine $N$.
        \begin{equation}\label{equation:N}
        N = S \times scope + \sum_{i=1}^R scope ^ {arity(r_i)}
        \end{equation}

    \end{Ventry}



    The combined complexity of all four steps is
    \begin{equation*}
        O(N) + O(nN) + O(N^2) + O(F)
    \end{equation*}
    Again, these steps are completed once for every breakpoint in the target program's
    execution; therefore, the overall upper bound becomes
    \begin{equation*}
    \begin{split}
        & b \times O(N) + b \times O(nN) + b \times O(N^2) + b \times  O(F) \\
                   = ~ & O(bN + bnN + bN^2 + bF)
    \end{split}
    \end{equation*}


    The vector [$x_0$  $x_1$] represents the two possible atoms of type
    \texttt{X}.  With our naming scheme, $x_0$ represents \texttt{X\_0} and $x_1$
    represents \texttt{X\_1}.
    The binary relation itself is represented by a two-dimensional bit matrix where a 1 in
    position [$i$,$j$] means that there is a mapping between the $i^{th}$ atom of \texttt{X}
    and the $j^{th}$ atom of \texttt{Y}:

    \begin{gather*}
    \begin{bmatrix}
      r_{00} & r_{01} \\
      r_{10} & r_{11} \\
    \end{bmatrix} \quad
    \begin{bmatrix}
      \texttt{X\_0->Y\_0} & \texttt{X\_0->Y\_1} \\
      \texttt{X\_1->Y\_0} & \texttt{X\_1->Y\_1} \\
    \end{bmatrix}
    \end{gather*}
    \medskip

    Now, consider a fact stating that relation \texttt{r} is total, i.e.,

    \begin{center}
    \texttt{all x :~X | some y :~Y | x.r = y}
    \end{center}

    The CNF formula for our example fact, in scope~2, is
    \begin{equation*}
    \begin{split}
    & \neg(((x_0 \wedge r_{00}) \vee (x_1 \wedge r_{10})) \wedge \neg((x_0 \wedge r_{01})
    \vee (x_1 \wedge r_{11}))) \wedge \\ &\neg(\neg((x_0 \wedge r_{00}) \vee (x_1 \wedge
    r_{10})) \wedge ((x_0 \wedge r_{01}) \vee (x_1 \wedge r_{11})))
    \end{split}
    \end{equation*}
    \medskip

    Table~\vref{fig:fullRunTimesAllPhases} contains...

    \begin{singlespacing}
    \begin{center}
    \begin{threeparttable}
    \caption{Running times for each phase and total running time of Embee}
    \label{fig:fullRunTimesAllPhases}\begin{small}
        % Table generated by Excel2LaTeX from sheet 'Embee'
        \begin{tabular}{|l|c|c|c|c|c|c|c|} \hline

        \multicolumn{3}{|c|}{Test Case} & \multicolumn{ 5}{c|}{Running Time (m:ss)} \\ \hline

        \multicolumn{1}{|c|}{Object} & & Number of & & & \multicolumn{2}{c|}{ Phase 3} &  \\  \cline{6-7}

        \multicolumn{1}{|c|}{Model} & \raisebox{1.5ex}[0pt]{Scope} & Breakpoints &  \raisebox{1.5ex}[0pt]{Phase 1} & \raisebox{1.5ex}[0pt]{Phase 2} &   First 16 &   Last 4 &  \raisebox{1.5ex}[0pt]{Total} \\

        \hline
            List  &  20 &  20           &  0:07 &  0:32 &  0:12 &  06:39 & 07:30 \\
        \hline
            Graph &  20 &  19\tnote{a}  &  0:07 &  1:27 &  0:35 &  44:10 & 46:19 \\
        \hline
            Tree  &  20 &  20 &  0:04   &  1:20 &  0:21 &  06:04 &  07:49 \\
        \hline
        \end{tabular}
    \begin{tablenotes}
    \item[a] Breakpoints occur after the addition of each edge, i.e., the first
    breakpoint does not occur until the second node is added.
    \end{tablenotes}
        \end{small}
    \end{threeparttable}
    \end{center}
    \end{singlespacing}


    ...upper bound on Embee's performance:

    \begin{numcases}{\text{upper bound is}}
        O(bN^2) & if $scope \leq 16$ \notag \\
        O(bF)   & if $scope > 16$ \notag
    \end{numcases}
