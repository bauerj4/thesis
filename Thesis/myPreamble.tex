% Queen's Thesis Format
% (Borrowed from Dean Jin's BigDis.tex file, then heavily modified :)

% Michelle L. Crane, Queen's University, 2003

%*************************************************************************************************************
% DOCUMENT STYLE
%*************************************************************************************************************
\documentclass[12pt]{report}
%-------------------------------------------------------------------------------------------------------------
\usepackage{quthesis}        % the Queen's University dissertation style file
                             % Note:  In my thesis, I had many Listings, so I
                             % tweaked the old quthesis sty file to create a
                             % List of Listings in the table of contents.
                             % However, this version of quthesis does *not*
                             % include these modifications.
%\usepackage[sectionbib]{natbib}
%\usepackage{chapterbib}  
\usepackage{chapterbib}

\usepackage{listings}
\usepackage{color}
\usepackage{longtable}


%I don't even use the fancyheadings - it looks nice enough without it
%\usepackage{fancyheadings}  % doesn't seem to change the headings at all!
%*************************************************************************************************************

\usepackage{hyperref}

%*************************************************************************************************************
% SPACING
%*************************************************************************************************************
\usepackage{setspace}        % for use of \singlespacing and \doublespacing
%*************************************************************************************************************


%*************************************************************************************************************
% HEADINGS
%*************************************************************************************************************

% This changes the headings go that they are prettier, this can be commented out for traditional headings.
%\usepackage{sectsty}
%\allsectionsfont{\bfseries}% set all the section font to bfseries
%\chapterfont{\centering\Large} % set the sizes of chapters, sections ...
%\sectionfont{\normalsize}
%\subsectionfont{\normalsize}

% for formatting Table of Contents entry, example: Chapter 1 Introduction
\usepackage[subfigure]{tocloft}
\usepackage{tocloft}
\renewcommand{\cftchappresnum}{Chapter }
\renewcommand{\cftchapaftersnum}{:}
\renewcommand{\cftchapnumwidth}{7em}

% for formatting Table of Contents entry for Appendix, example: Appendix 1: Stuff
\newcommand*\updatechaptername{%
   \addtocontents{toc}{\protect\renewcommand*\protect\cftchappresnum{Appendix }}
}

%*************************************************************************************************************
% FOOTNOTES
%*************************************************************************************************************

\interfootnotelinepenalty=10000 % This line stops footnotes from splitting onto two pages.

%*************************************************************************************************************
% VERBATIM
%*************************************************************************************************************
\usepackage{moreverb}        % Using this package to get better control of the
                             % verbatim environment, mostly for the use of the
                             % listing environment which puts line number
                             % beside each line.  Note that there has to be a number
                             % in each set of brackets, i.e., \begin{listing}[1]{1}.
                             % PDF info file is "The moreverb package" by
                             % Robin Fairbairns (rf@cl.cam.ac.uk) after
                             % Angus Duggan, Rainer Schopf and Victor Eijkhout, 2000/06/29.
%-------------------------------------------------------------------------------------------------------------
%\usepackage{verbatim}        % allows the use of \begin{comment} and \end{comment}
                             % as well as \verbatiminput{file}
                             % Note:  when using verbatim to input from a text file,
                             % such as a specification or code, use \begin{singlespacing}
                             % and \end{singlespacing}.  Also, tabs are not read
                             % properly, so the input file must only use spaces.

%                             \begin{comment}
%                             Can also use the verbatim package for
%                             comments like this...
%                             \end{comment}
%*************************************************************************************************************


%*************************************************************************************************************
% GLOSSARY
% Using a glossary is more than beginners need to know; leaving the packages, etc. here for now.
%*************************************************************************************************************
%\usepackage[nonumberlist]{glossaries} % use glossaries since glossary package is out dated
%\makeglossaries  % tell latex to make the glossary
%\glossarystyle{list}

%*************************************************************************************************************


%*************************************************************************************************************
% INDEX
% Also possible to make an index; didn't use for my thesis.
%*************************************************************************************************************
%\usepackage{makeidx}         % to make the index
%-------------------------------------------------------------------------------------------------------------
% Tell Latex to make an index
%\makeindex
%*************************************************************************************************************


%*************************************************************************************************************
% MATH STUFF
%*************************************************************************************************************
\usepackage{amsmath}         % to make nice equations
\usepackage{amssymb} 

%-------------------------------------------------------------------------------------------------------------
\usepackage{amsthm}          % to make nice theorem, i.e., definition

% Using the amsthm package, define a new theorem environment for my
% definition.  * means don't number it.
\newtheorem*{definition}{Definition}
%-------------------------------------------------------------------------------------------------------------
\usepackage{cases}           % to make numbered cases (equations)
%-------------------------------------------------------------------------------------------------------------
\usepackage{calc}            % Used with the Ventry environment defined below.
%*************************************************************************************************************


%*************************************************************************************************************
% FLOATS AND FIGURES
%*************************************************************************************************************
\usepackage{graphicx}        % for graphic images (use \includegraphics[...]{file.eps})
%-------------------------------------------------------------------------------------------------------------
\usepackage{subfigure}       % for subfigures (figures within figures)
%-------------------------------------------------------------------------------------------------------------
\usepackage{boxedminipage}   % to make boxed minipages, i.e., boxes around figures
%-------------------------------------------------------------------------------------------------------------
\usepackage{rotate}          % for use of \begin{sideways} and \end{sideways}
%-------------------------------------------------------------------------------------------------------------
\usepackage{float}           % Using this package to get better control of my floats
                             % including the ability to define new float types for
                             % my specification and code listings.
                             % DVI info file is "An Improved Environment for Floats"
                             % by Anselm Lingnau, lingnau@tm.informatik.uni=frankfurt.de
                             % 1995/03/29.

% Define new float styles here
% Ruled style for examples
%\floatstyle{ruled}
%\newfloat{Example}{h}{lop}[chapter]

% Style of float used for code listings
\floatstyle{ruled}
\newfloat{Listing}{H}{lis}[chapter]

                             % Note:  The listings don't have space between the chapters, unlike
                             % the standard list of tables etc.  At the end, copy the spacing
                             % commands from the .toc file and insert into the .lis file.  Then,
                             % DO NOT LATEX it again, simply go to the DVI viewer!
%*************************************************************************************************************
% TABLES
%*************************************************************************************************************
\usepackage{tabularx}        % Package used to make variable width-columns, i.e.,
                             % column widths are changed to fit the maximum width
                             % and text is wrapped nicely.

\usepackage{threeparttable}
%*************************************************************************************************************
% CAPTIONS
%*************************************************************************************************************
\usepackage[hang]{caption}   % Package used to make my captions 'hang', i.e., wrap
                             % around, but not under the name of the caption.
%-------------------------------------------------------------------------------------------------------------
% Find that the captions are too far from my verbatim figures, but if
% I change it to 0, then the captions are too close for my other types
% of figures.  Maybe set each one separately?
%\setlength{\abovecaptionskip}{1ex}

%\setlength{\textfloatsep}{1ex plus1pt minus1pt}

%\setlength{\intextsep}{1ex plus1pt minus1pt}

%\setlength{\floatsep}{1ex plus1pt minus1pt}
%*************************************************************************************************************


%*************************************************************************************************************
% MISCELLANEOUS
%*************************************************************************************************************
\usepackage{layout}          % useful for determining the margins of a document
                             % use with \layout command
%-------------------------------------------------------------------------------------------------------------
\usepackage{changebar}       % Way of indicating modifications by putting bars in the
                             % margin.  Read about it in "The Latex Companion".
%*************************************************************************************************************


%*************************************************************************************************************
% REFERENCES ETC.
%*************************************************************************************************************
\usepackage{varioref}        % Better page references, e.g., "on preceding page", etc.
                             % \vref{key} Create an enhanced reference.
                             % \vpageref[text]{key} Create an enhanced page reference.
                             % \vrefrange{key}{key} Create an enhanced range of references.
                             % \vpagerefrange[text]{key}{key} Create an enhanced range of page references.
                             % Note: doesn't really work for consecutive pages.

% Renewing the text for before and after, because I don't like the default flip-flopping one.
% And 'on the page before' sounds dumb!

\renewcommand{\reftextafter}{on the next page}
\renewcommand{\reftextbefore}{on the previous page}
%-------------------------------------------------------------------------------------------------------------
\usepackage{url}             % for use of \url - pretty web addresses
%*************************************************************************************************************
% HYPERLINKS (must be last)
%*************************************************************************************************************
%\usepackage[]{hyperref}
%\usepackage[dvips,bookmarks]{hyperref}
                             % Neat package to turn href, ref, cite, gloss entries
                             % into hyperlinks in the dvi file.
                             % Make sure this is the last package loaded.
                             % Use with dvips option to get hyperlinks to work in ps and pdf
                             % files.  Unfortunately, then they don't work in the dvi file!
                             % Use without the dvips option to get the links to work in the dvi file.
                             % Note:  \floatstyle{ruled} don't work properly; so change to plain.
                             % Not as pretty, but functional...
                             % The bookmarks option sets up proper bookmarks in the pdf file :)

% Need this command to allow hyperref to play nicely with gloss; otherwise
% almost every \gloss will cause an error...
%\renewcommand{\glosslinkborder}{0 0 0}
%*************************************************************************************************************


%*************************************************************************************************************
% MISCELLANEOUS COMMANDS AND ENVIRONMENTS
%*************************************************************************************************************
% Use this command to show more table of contents - used when playing
% with the draft outline
% I think it should be about 2???
\setcounter{tocdepth}{2}
%*************************************************************************************************************
% Environment definition I found in the "The Latex Companion".  Used to
% create a list environment where the indenting is the same for all of the
% entries, regardless of their length.  Note:  must \usepackage{calc}.
\newenvironment{Ventry}[1]%
    {\begin{list}{}{\renewcommand{\makelabel}[1]{\textbf{##1}\hfil}%
        \settowidth{\labelwidth}{\textbf{#1:}}%
        \setlength{\leftmargin}{\labelwidth+\labelsep}}}%
    {\end{list}}
%*************************************************************************************************************

%*************************************************************************************************************
% MY DEFINED COMMANDS
%*************************************************************************************************************
% Command that I can use to create notes in the margins;
% adapted from Juergen's META tag
%\newcommand{\meta}[1]{\begin{singlespacing}
%{\marginpar{\emph{\footnotesize Note: #1}}}\end{singlespacing}}
%*************************************************************************************************************
% Command that I can use to create lined headings
%\newcommand{\heading}[1]{\bigskip \hrule \smallskip \noindent \texttt{#1} \smallskip \hrule}
%*************************************************************************************************************
% Command that I can use for reading in a file, verbatim, with line
% numbers printed along the left side.  The parameter is the file name.
%\newcommand{\fileinnum}[1]{
%    \begin{singlespacing} {\footnotesize
%    \begin{listinginput}[1]{1}{#1}\end{listinginput}
%    }\end{singlespacing}
%}
%*************************************************************************************************************
% Command that I can use for reading in a file, verbatim, with NO line
% numbers, but in a smaller font.  The parameter is the file name.
\newcommand{\filein}[1]{
   \begin{singlespacing}{\footnotesize
    \begin{verbatiminput}{#1}\end{verbatiminput}
    }\end{singlespacing}
}
%*************************************************************************************************************
% Command that I can use for reading in a file, verbatim, with NO line
% numbers, but in a smaller font.  The parameter is the file name.
\newcommand{\fileinsmall}[1]{
    \begin{singlespacing}{\scriptsize
    \begin{verbatiminput}{#1}\end{verbatiminput}
    }\end{singlespacing}
}
%*************************************************************************************************************
% Dean't 'notesbox' command.  Needs setspace package.
%   Usage: \notesbox{This is a note.}
%%
\newcommand{\notesbox}[1]{
     \ \\
      \singlespacing
      \noindent\begin{boxedminipage}[h]{\textwidth}{\sf{#1}}\end{boxedminipage}
      \doublespacing
}


% Standard journal abbreviations
% Mostly as used by ADS, with a few additions for journals where MNRAS does not
% follow normal IAU style.

\newcommand\aap{A\&A}                % Astronomy and Astrophysics
\let\astap=\aap                          % alternative shortcut
\newcommand\aapr{A\&ARv}             % Astronomy and Astrophysics Review (the)
\newcommand\aaps{A\&AS}              % Astronomy and Astrophysics Supplement Series
\newcommand\actaa{Acta Astron.}      % Acta Astronomica
\newcommand\afz{Afz}                 % Astrofizika
\newcommand\aj{AJ}                   % Astronomical Journal (the)
\newcommand\ao{Appl. Opt.}           % Applied Optics
\let\applopt=\ao                         % alternative shortcut
\newcommand\aplett{Astrophys.~Lett.} % Astrophysics Letters
\newcommand\apj{ApJ}                 % Astrophysical Journal
\newcommand\apjl{ApJ}                % Astrophysical Journal, Letters
\let\apjlett=\apjl                       % alternative shortcut
\newcommand\apjs{ApJS}               % Astrophysical Journal, Supplement
\let\apjsupp=\apjs                       % alternative shortcut
% The following journal does not appear to exist! Disabled.
%\newcommand\apspr{Astrophys.~Space~Phys.~Res.} % Astrophysics Space Physics Research
\newcommand\apss{Ap\&SS}             % Astrophysics and Space Science
\newcommand\araa{ARA\&A}             % Annual Review of Astronomy and Astrophysics
\newcommand\arep{Astron. Rep.}       % Astronomy Reports
\newcommand\aspc{ASP Conf. Ser.}     % ASP Conference Series
\newcommand\azh{Azh}                 % Astronomicheskii Zhurnal
\newcommand\baas{BAAS}               % Bulletin of the American Astronomical Society
\newcommand\bac{Bull. Astron. Inst. Czechoslovakia} % Bulletin of the Astronomical Institutes of Czechoslovakia 
\newcommand\bain{Bull. Astron. Inst. Netherlands} % Bulletin Astronomical Institute of the Netherlands
\newcommand\caa{Chinese Astron. Astrophys.} % Chinese Astronomy and Astrophysics
\newcommand\cjaa{Chinese J.~Astron. Astrophys.} % Chinese Journal of Astronomy and Astrophysics
\newcommand\fcp{Fundamentals Cosmic Phys.}  % Fundamentals of Cosmic Physics
\newcommand\gca{Geochimica Cosmochimica Acta}   % Geochimica Cosmochimica Acta
\newcommand\grl{Geophys. Res. Lett.} % Geophysics Research Letters
\newcommand\iaucirc{IAU~Circ.}       % IAU Cirulars
\newcommand\icarus{Icarus}           % Icarus
\newcommand\japa{J.~Astrophys. Astron.} % Journal of Astrophysics and Astronomy
\newcommand\jcap{J.~Cosmology Astropart. Phys.} % Journal of Cosmology and Astroparticle Physics
\newcommand\jcp{J.~Chem.~Phys.}      % Journal of Chemical Physics
\newcommand\jgr{J.~Geophys.~Res.}    % Journal of Geophysics Research
\newcommand\jqsrt{J.~Quant. Spectrosc. Radiative Transfer} % Journal of Quantitiative Spectroscopy and Radiative Transfer
\newcommand\jrasc{J.~R.~Astron. Soc. Canada} % Journal of the RAS of Canada
\newcommand\memras{Mem.~RAS}         % Memoirs of the RAS
\newcommand\memsai{Mem. Soc. Astron. Italiana} % Memoire della Societa Astronomica Italiana
\newcommand\mnassa{MNASSA}           % Monthly Notes of the Astronomical Society of Southern Africa
\newcommand\mnras{MNRAS}             % Monthly Notices of the Royal Astronomical Society
\newcommand\na{New~Astron.}          % New Astronomy
\newcommand\nar{New~Astron.~Rev.}    % New Astronomy Review
\newcommand\nat{Nature}              % Nature
\newcommand\nphysa{Nuclear Phys.~A}  % Nuclear Physics A
\newcommand\pra{Phys. Rev.~A}        % Physical Review A: General Physics
\newcommand\prb{Phys. Rev.~B}        % Physical Review B: Solid State
\newcommand\prc{Phys. Rev.~C}        % Physical Review C
\newcommand\prd{Phys. Rev.~D}        % Physical Review D
\newcommand\pre{Phys. Rev.~E}        % Physical Review E
\newcommand\prl{Phys. Rev.~Lett.}    % Physical Review Letters
\newcommand\pasa{Publ. Astron. Soc. Australia}  % Publications of the Astronomical Society of Australia
\newcommand\pasp{PASP}               % Publications of the Astronomical Society of the Pacific
\newcommand\pasj{PASJ}               % Publications of the Astronomical Society of Japan
\newcommand\physrep{Phys.~Rep.}      % Physics Reports
\newcommand\physscr{Phys.~Scr.}      % Physica Scripta
\newcommand\planss{Planet. Space~Sci.} % Planetary Space Science
\newcommand\procspie{Proc.~SPIE}     % Proceedings of the Society of Photo-Optical Instrumentation Engineers
\newcommand\rmxaa{Rev. Mex. Astron. Astrofis.} % Revista Mexicana de Astronomia y Astrofisica
\newcommand\qjras{QJRAS}             % Quarterly Journal of the RAS
\newcommand\sci{Science}             % Science
\newcommand\skytel{Sky \& Telesc.}   % Sky and Telescope
\newcommand\solphys{Sol.~Phys.}      % Solar Physics
\newcommand\sovast{Soviet~Ast.}      % Soviet Astronomy (aka Astronomy Reports)
\newcommand\ssr{Space Sci. Rev.}     % Space Science Reviews
\newcommand\zap{Z.~Astrophys.}       % Zeitschrift fuer Astrophysik



% Extra includes 

\usepackage{natbib}
\usepackage[outdir=./]{epstopdf}